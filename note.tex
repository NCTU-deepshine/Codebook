\section{歐拉定理}
假若 $a$ 與 $n$ 互質,那麼 $a^{\phi(x)}-1$ 可被 $n$ 整除。亦即,$a^{\phi(n)}\equiv1(\mbox{ mod }n)$。
$\phi(n)=\phi(p^k)=p^k-p^{k-1}=(p-1)p^{k-1}$。
若$m,n$互質,則$\phi(mn)=\phi(m)\phi(n)$。

\section{路卡斯公式}
\[{m\choose n}\equiv\prod_{i=0}^k{m_i \choose n_i}(\mbox{mod }p)\] where $m=m_kp^k+m_{k-1}p^{k-1}+\cdots+m_1p+m_0$ and 
$n=n_kp^k+n_{k-1}p^{k-1}+\cdots+n_1p+n_0$.

\section{模數合成}
見 codeBook: modCombine

\section{強國人說的歐拉定理}
如果 $a$ 和 $n$ 互質,那麼 $a^{\phi(n)}\equiv1(\mbox{mod }n)$,對於任意 $a,n$和較大的 $b$,有 $a^b\equiv a^{\phi(n)+b\mbox{ mod }\phi(n)}(\mbox{ mod }n)$

\section{無權邊的生成樹個數 Kirchhoff's Theorem}
1. 定義 $n\times m$矩陣 $E=(a_{i,j})$,$n$ 為點數, $m$ 為邊數,若 $i$ 點在 $j$ 編上,$i$為小點 $a_{i,j}=1$, $i$為大點 $a_{i,j}=-1$,否則 $a_{i,j}=0$。\\
(證明省略)
\\
4.令 $E(E^T)=Q$,他是一種有負號的 kirchhoff 的矩陣,取 $Q$ 的子矩陣即為 $F(F^T)$\\
結論:做 $Q$ 取子矩陣算 det 即為所求。(除去第一行第一列 by mz)

\section{很大的質數}
18446744082299486207

\section{GP 東北數學式}
$(p-1)!/p \%p = p-1$\\
$C(n,m) = C(n/p,m/p) * C(n\%p,m\%p)$

\section{圓周率1000位}
3.141592653589793238462643383279502884197169399375105820974944592307816406286\\
20899862803482534211706798214808651328230664709384460955058223172535940812848\\
11174502841027019385211055596446229489549303819644288109756659334461284756482\\
33786783165271201909145648566923460348610454326648213393607260249141273724587\\
00660631558817488152092096282925409171536436789259036001133053054882046652138\\
41469519415116094330572703657595919530921861173819326117931051185480744623799\\
62749567351885752724891227938183011949129833673362440656643086021394946395224\\
73719070217986094370277053921717629317675238467481846766940513200056812714526\\
35608277857713427577896091736371787214684409012249534301465495853710507922796\\
89258923542019956112129021960864034418159813629774771309960518707211349999998\\
37297804995105973173281609631859502445945534690830264252230825334468503526193\\
11881710100031378387528865875332083814206171776691473035982534904287554687311\\
59562863882353787593751957781857780532171226806613001927876611195909216420199  

\section{尤拉數$e$}
2.718281828459045235360287471352662497757247093699959574966967627724076630353\\
54759457138217852516642742746

\section{歐拉示性數}
$\chi=F-E+V$\\
幾何中同類的形狀,$\chi$為相同值

\section{半平面交相關幾何轉換}
$(a,b)\Leftrightarrow y=ax+b$

