%!TEX TS-program = xelatex
%!TEX encoding = UTF-8 Unicode
\documentclass [landscape,10pt,a4paper,twocolumn,nofonts]{article}
\title {NCTU electron Codebook}
\usepackage{parskip}
\usepackage{xeCJK} 
\setmainfont{Ubuntu Mono}
\setCJKmainfont{WenQuanYi Micro Hei}
\usepackage {listings}
\usepackage {color}
\usepackage [left=1.2cm, right=1.2cm, top=2.2cm, bottom=1.2cm]{geometry}
\definecolor {mygreen}{rgb}{0,0.6,0}
\definecolor {mygray}{rgb}{0.5,0.5,0.5}
\definecolor {mymauve}{rgb}{0.58,0,0.82}
\usepackage{fancyheadings}
\rhead{\thepage}
\lhead{National Chiao Tung University, NCTU\_electron}
\pagestyle{fancy}
\cfoot{}

\lstset { %
  backgroundcolor=\color{white},   % choose the background color; you must add \usepackage{color} or \usepackage{xcolor}
  basicstyle=\small\ttfamily,               % the size of the fonts that are used for the code
  breakatwhitespace=false,         % sets if automatic breaks should only happen at whitespace
  breaklines=true,                 % sets automatic line breaking
  captionpos=b,                    % sets the caption-position to bottom
  commentstyle=\color{mygreen},    % comment style
  deletekeywords={...},            % if you want to delete keywords from the given language
  escapeinside={\%*}{*)},          % if you want to add LaTeX within your code
  extendedchars=true,              % lets you use non-ASCII characters; for 8-bits encodings only, does not work with UTF-8
  frame=single,                    % adds a frame around the code
  keepspaces=true,                 % keeps spaces in text, useful for keeping indentation of code (possibly needs columns=flexible)
  keywordstyle=\color{blue},       % keyword style
  language=Octave,                 % the language of the code
  morekeywords={*,...},            % if you want to add more keywords to the set
  numbers=none,                    % where to put the line-numbers; possible values are (none, left, right)
  numbersep=4pt,                   % how far the line-numbers are from the code
  numberstyle=\tiny\color{mygray}, % the style that is used for the line-numbers
  rulecolor=\color{black},         % if not set, the frame-color may be changed on line-breaks within not-black text (e.g. comments (green here))
  showspaces=false,                % show spaces everywhere adding particular underscores; it overrides 'showstringspaces'
  showstringspaces=false,          % underline spaces within strings only
  showtabs=false,                  % show tabs within strings adding particular underscores
  stepnumber=2,                    % the step between two line-numbers. If it's 1, each line will be numbered
  stringstyle=\color{mymauve},     % string literal style
  tabsize=2,                       % sets default tabsize to 2 spaces
  %title=\lstname                   % show the filename of files included with \lstinputlisting; also try caption instead of title
}
\begin {document}
\thispagestyle{fancy}
%\maketitle
\tableofcontents
\newpage
\subsection{.vimrc}
\lstinputlisting [language=bash] {code/.vimrc}
\subsection{AC Actomaton}
\lstinputlisting [language=c++] {code/AC_Automaton.cpp}
\subsection{Combinatoion}
\lstinputlisting [language=c++] {code/combination.cpp}
%\lstinputlisting [language=java] {code/ConvexHull.java}
\subsection{Decomposition}
\lstinputlisting [language=java] {code/Decomposition.java}
\subsection{Double LCA}
\lstinputlisting [language=c++] {code/double_lca.cpp}
\subsection{Flow (Dinics)}
\lstinputlisting [language=java] {code/Flow_Dinics.java}
\subsection{Geometry}
\lstinputlisting [language=c++] {code/Geometry.cpp}
%\lstinputlisting [language=c++] {code/Hash.cpp}
\subsection{Simple Tabulation Hash}
\lstinputlisting [language=java] {code/SimpleTabulationHash.java}
\subsection{IDA*}
\lstinputlisting [language=c++] {code/IDAstar.cpp}
\subsection{inverse}
\lstinputlisting [language=c++] {code/inverse.cpp}
\subsection{Karatsuba (FFFT)}
\lstinputlisting [language=java] {code/Karatsuba.java}
\subsection{KM}
\lstinputlisting [language=c++] {code/km.cpp}
\subsection{Linear Prime}
\lstinputlisting [language=c++] {code/linear_prime.cpp}
%\lstinputlisting [language=java] {code/MyHashMap.java}
\subsection{Max clique}
\lstinputlisting [language=java] {code/MaxClique.java}
\subsection{Mod Combine}
\lstinputlisting [language=c++] {code/modCombine.cpp}
\subsection{Range Tree 2D, kth number}
\lstinputlisting [language=c++] {code/rangeTree2D.cpp}
\subsection{Range Tree 2D, rectangle}
\lstinputlisting [language=c++] {code/rangeTree2D_rq.cpp}
\subsection{Scan (JAVA)}
\lstinputlisting [language=java] {code/Scan.java}
%\lstinputlisting [language=java] {code/SegmentTree_Basic.java}
%\lstinputlisting [language=java] {code/SegmentTree.java}
\subsection{Segment Tree}
\lstinputlisting [language=c++] {code/SegmentTree.cpp}
\subsection{Splay Tree}
\lstinputlisting [language=java] {code/SplayTree.java}
%\lstinputlisting [language=c++] {code/SuffixArray.cpp}
\subsection{Suffix Array}
\lstinputlisting [language=java] {code/SuffixArray.java}
%\lstinputlisting [language=c++] {code/tarjan_lca.cpp}
%\lstinputlisting [language=java] {code/Tester.java}
\subsection{Treap}
\lstinputlisting [language=c++] {code/treap.cpp}
%\lstinputlisting [language=java] {code/Treap.java}
\subsection{Z Algorithm}
\lstinputlisting [language=c++] {code/z_algorithm.cpp}
\newpage
\section{歐拉定理}
假若 $a$ 與 $n$ 互質,那麼 $a^{\phi(x)}-1$ 可被 $n$ 整除。亦即,$a^{\phi(n)}\equiv1(\mbox{ mod }n)$。
$\phi(n)=\phi(p^k)=p^k-p^{k-1}=(p-1)p^{k-1}$。
若$m,n$互質,則$\phi(mn)=\phi(m)\phi(n)$。

\section{路卡斯公式}
\[{m\choose n}\equiv\prod_{i=0}^k{m_i \choose n_i}(\mbox{mod }p)\] where $m=m_kp^k+m_{k-1}p^{k-1}+\cdots+m_1p+m_0$ and 
$n=n_kp^k+n_{k-1}p^{k-1}+\cdots+n_1p+n_0$.

\section{模數合成}
見 codeBook: modCombine

\section{強國人說的歐拉定理}
如果 $a$ 和 $n$ 互質,那麼 $a^{\phi(n)}\equiv1(\mbox{mod }n)$,對於任意 $a,n$和較大的 $b$,有 $a^b\equiv a^{\phi(n)+b\mbox{ mod }\phi(n)}(\mbox{ mod }n)$

\section{無權邊的生成樹個數 Kirchhoff's Theorem}
1. 定義 $n\times m$矩陣 $E=(a_{i,j})$,$n$ 為點數, $m$ 為邊數,若 $i$ 點在 $j$ 編上,$i$為小點 $a_{i,j}=1$, $i$為大點 $a_{i,j}=-1$,否則 $a_{i,j}=0$。\\
(證明省略)
\\
4.令 $E(E^T)=Q$,他是一種有負號的 kirchhoff 的矩陣,取 $Q$ 的子矩陣即為 $F(F^T)$\\
結論:做 $Q$ 取子矩陣算 det 即為所求。(除去第一行第一列 by mz)

\section{很大的質數}
18446744082299486207

\section{GP 東北數學式}
$(p-1)!/p \%p = p-1$\\
$C(n,m) = C(n/p,m/p) * C(n\%p,m\%p)$

\section{圓周率1000位}
3.141592653589793238462643383279502884197169399375105820974944592307816406286\\
20899862803482534211706798214808651328230664709384460955058223172535940812848\\
11174502841027019385211055596446229489549303819644288109756659334461284756482\\
33786783165271201909145648566923460348610454326648213393607260249141273724587\\
00660631558817488152092096282925409171536436789259036001133053054882046652138\\
41469519415116094330572703657595919530921861173819326117931051185480744623799\\
62749567351885752724891227938183011949129833673362440656643086021394946395224\\
73719070217986094370277053921717629317675238467481846766940513200056812714526\\
35608277857713427577896091736371787214684409012249534301465495853710507922796\\
89258923542019956112129021960864034418159813629774771309960518707211349999998\\
37297804995105973173281609631859502445945534690830264252230825334468503526193\\
11881710100031378387528865875332083814206171776691473035982534904287554687311\\
59562863882353787593751957781857780532171226806613001927876611195909216420199  

\section{尤拉數$e$}
2.718281828459045235360287471352662497757247093699959574966967627724076630353\\
54759457138217852516642742746

\section{歐拉示性數}
$\chi=F-E+V$\\
幾何中同類的形狀,$\chi$為相同值

\section{半平面交相關幾何轉換}
$(a,b)\Leftrightarrow y=ax+b$


\end {document}
